\PassOptionsToPackage{unicode=true}{hyperref} % options for packages loaded elsewhere
\PassOptionsToPackage{hyphens}{url}
%
\documentclass[]{article}
\usepackage{lmodern}
\usepackage{amssymb,amsmath}
\usepackage{ifxetex,ifluatex}
\usepackage{fixltx2e} % provides \textsubscript
\ifnum 0\ifxetex 1\fi\ifluatex 1\fi=0 % if pdftex
  \usepackage[T1]{fontenc}
  \usepackage[utf8]{inputenc}
  \usepackage{textcomp} % provides euro and other symbols
\else % if luatex or xelatex
  \usepackage{unicode-math}
  \defaultfontfeatures{Ligatures=TeX,Scale=MatchLowercase}
\fi
% use upquote if available, for straight quotes in verbatim environments
\IfFileExists{upquote.sty}{\usepackage{upquote}}{}
% use microtype if available
\IfFileExists{microtype.sty}{%
\usepackage[]{microtype}
\UseMicrotypeSet[protrusion]{basicmath} % disable protrusion for tt fonts
}{}
\IfFileExists{parskip.sty}{%
\usepackage{parskip}
}{% else
\setlength{\parindent}{0pt}
\setlength{\parskip}{6pt plus 2pt minus 1pt}
}
\usepackage{hyperref}
\hypersetup{
            pdfborder={0 0 0},
            breaklinks=true}
\urlstyle{same}  % don't use monospace font for urls
\setlength{\emergencystretch}{3em}  % prevent overfull lines
\providecommand{\tightlist}{%
  \setlength{\itemsep}{0pt}\setlength{\parskip}{0pt}}
\setcounter{secnumdepth}{0}
% Redefines (sub)paragraphs to behave more like sections
\ifx\paragraph\undefined\else
\let\oldparagraph\paragraph
\renewcommand{\paragraph}[1]{\oldparagraph{#1}\mbox{}}
\fi
\ifx\subparagraph\undefined\else
\let\oldsubparagraph\subparagraph
\renewcommand{\subparagraph}[1]{\oldsubparagraph{#1}\mbox{}}
\fi

% set default figure placement to htbp
\makeatletter
\def\fps@figure{htbp}
\makeatother


\date{}

\begin{document}

\hypertarget{header-n3}{%
\subsection{Inducción}\label{header-n3}}

\hypertarget{header-n8}{%
\subparagraph{\texorpdfstring{1. Demuestra por inducción sobre
\(n \ge 0\), que
\(\sum_{i=1}^{n}{i(i+1)=\frac{n(n+1)(n+2)}{3}}\).}{1. Demuestra por inducción sobre n \textbackslash{}ge 0, que \textbackslash{}sum\_\{i=1\}\^{}\{n\}\{i(i+1)=\textbackslash{}frac\{n(n+1)(n+2)\}\{3\}\}.}}\label{header-n8}}

Tenemos que \(n \in \N\), vamos a probar para los siguientes casos:

\begin{itemize}
\item
  Cuando \(n = 0\)

  \[\sum_{i=1}^{n}{i(i+1)=0(0+1)= \frac{0(0+1)(0+2)}{3}} = 0\]

  Por lo tanto cumple cuando \(n = 0\).
\end{itemize}

\begin{itemize}
\item
  Cuando \(n = 1\)

  \[\sum_{i=1}^{n}{i(i+1)=0 + 1(1+1)= \frac{1(1+1)(1+2)}{3}} = 2\]
\end{itemize}

Hipótesis inductiva: supongamos que el teorema se cumple para todos los
valores de \(n\) ahora demostremos para \(n+1\).

\begin{equation}
\begin{split}
{\sum_{i=1}^{n+1}{i(i+1)}} & = (n+1)((n+1)+1) + {\sum_{i=1}^{n}{i(i+1)}} \\
& = \frac{n(n+1)(n+2)}{3} + (n+1)((n+1)+1)\ \text {, por nuestra hipótesis de inducción.}\\
& = \frac{n(n+1)(n+2)}{3}+ \frac{3(n+1)((n+1)+1)}{3} \\
& = \frac{n(n+1)(n+2)+3(n+1)((n+1)+1)}{3} \\
& = \frac{(n+1)(n(n+2)+3((n+1) +1))}{3} \\
& = \frac{(n+1)(n(n+2)+3(n+2))}{3} \\
& = \frac{(n+1)(n^2+2n+3n+2)}{3} \\
& = \frac{(n+1)(n+2)(n+3))}{3} \\
& = \frac{(n+1)((n+1)+1)((n+1)+2))}{3} \\
\end{split}
\end{equation}

La equivalencia se cumple para \(n+1\).

Por el principio de inducción matemática, el teorema es válido para todo
\(n \in \N\)

\hypertarget{header-n11}{%
\subparagraph{\texorpdfstring{2. Demuestra por inducción sobre
\(n\ge 0\), que \(\sum_{i=1}^{n}{i^3}= \frac{n^2(n+1)^2}{4}\)
.}{2. Demuestra por inducción sobre n\textbackslash{}ge 0, que \textbackslash{}sum\_\{i=1\}\^{}\{n\}\{i\^{}3\}= \textbackslash{}frac\{n\^{}2(n+1)\^{}2\}\{4\} .}}\label{header-n11}}

Tenemos que \(n \in \N\), vamos a probar para los siguientes casos:

\begin{itemize}
\item
  Cuando \(n = 1\)

  \[\sum_{i=1}^{n}{i^3}= 1^3=\frac{1^2(1+1)^2}{4} = 1\]
\item
  Cuando \(n = 2\)

  \[\sum_{i=1}^{n}{i^3}= 1^3 + 2^3=\frac{2^2(2+1)^2}{4} = 9\]
\end{itemize}

Hipótesis inductiva: supongamos que el teorema se cumple para todos los
valores de \(n\) ahora demostremos para \(n = n+1\).

\begin{equation}
\begin{split}
\sum_{i=1}^{n+1}{i^3} & = (n+1)^3 + \sum_{i=1}^{n}{i^3} \\
& = \frac{n^2(n+1)^2}{4} + (n+1)^3 \ \text {, por nuestra hipótesis de inducción.} \\ 
& = \frac{n^2(n+1)^2 + 4(n+1)^3}{4} \\
& = \frac{(n+1)^2(n^2+4(n+1))}{4} \\
& = \frac{(n+1)^2(n^2+4n+4)}{4} \\
& = \frac{(n+1)^2(n+2)^2}{4} \\
& = \frac{(n+1)^2((n+1)+1)^2}{4}
\end{split}
\end{equation}

La equivalencia se cumple para \(n+1\).

Por el principio de inducción matemática, el teorema es válido para todo
\(n \in \N\)

\hypertarget{header-n7}{%
\subparagraph{\texorpdfstring{3. Demuestre por inducción sobre
\(n\ge5\), que
\(2^n > n^2\).}{3. Demuestre por inducción sobre n\textbackslash{}ge5, que 2\^{}n \textgreater{} n\^{}2.}}\label{header-n7}}

Tenemos que \(n \in \N\), vamos a probar para los siguientes casos:

\begin{itemize}
\item
  Cuando \(n = 5\)

  \[2^5 > n^2 \\
  32 > 25\]
\item
  Cuando \(n = 6\)

  \[2^6 > 6^2 \\
  64 > 36\]
\end{itemize}

Hipótesis inductiva: supongamos que el teorema se cumple para todos los
valores de \(n\) ahora demostremos para \(n = n+1\).

\[2 \cdot 2^{n} > (n+1)^2 \\
2^{(n+1)} > (n+1)^2\]

Por lo tanto se cumple \(\forall n \ge 5 \).

\end{document}
