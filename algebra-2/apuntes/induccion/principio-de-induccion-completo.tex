\PassOptionsToPackage{unicode=true}{hyperref} % options for packages loaded elsewhere
\PassOptionsToPackage{hyphens}{url}
%
\documentclass[]{article}
\usepackage{lmodern}
\usepackage{amssymb,amsmath}
\usepackage{ifxetex,ifluatex}
\usepackage{fixltx2e} % provides \textsubscript
\ifnum 0\ifxetex 1\fi\ifluatex 1\fi=0 % if pdftex
  \usepackage[T1]{fontenc}
  \usepackage[utf8]{inputenc}
  \usepackage{textcomp} % provides euro and other symbols
\else % if luatex or xelatex
  \usepackage{unicode-math}
  \defaultfontfeatures{Ligatures=TeX,Scale=MatchLowercase}
\fi
% use upquote if available, for straight quotes in verbatim environments
\IfFileExists{upquote.sty}{\usepackage{upquote}}{}
% use microtype if available
\IfFileExists{microtype.sty}{%
\usepackage[]{microtype}
\UseMicrotypeSet[protrusion]{basicmath} % disable protrusion for tt fonts
}{}
\IfFileExists{parskip.sty}{%
\usepackage{parskip}
}{% else
\setlength{\parindent}{0pt}
\setlength{\parskip}{6pt plus 2pt minus 1pt}
}
\usepackage{hyperref}
\hypersetup{
            pdfborder={0 0 0},
            breaklinks=true}
\urlstyle{same}  % don't use monospace font for urls
\setlength{\emergencystretch}{3em}  % prevent overfull lines
\providecommand{\tightlist}{%
  \setlength{\itemsep}{0pt}\setlength{\parskip}{0pt}}
\setcounter{secnumdepth}{0}
% Redefines (sub)paragraphs to behave more like sections
\ifx\paragraph\undefined\else
\let\oldparagraph\paragraph
\renewcommand{\paragraph}[1]{\oldparagraph{#1}\mbox{}}
\fi
\ifx\subparagraph\undefined\else
\let\oldsubparagraph\subparagraph
\renewcommand{\subparagraph}[1]{\oldsubparagraph{#1}\mbox{}}
\fi

% set default figure placement to htbp
\makeatletter
\def\fps@figure{htbp}
\makeatother


\date{}

\begin{document}

\textbf{Principio de inducción matemático, \(PI\)}

Sea \(P(n)\) un enunciado en el dominio de los números naturales.\\
 i. \(P(0)\) es verdadera y \\
 ii. Para todo \(n\in \N\), \(P(n)\) es verdadera, entonces \(P(n+1)\)
es verdadera

Entonces \(P(n)\) es verdadera parar todo \(n \in \N\)

\textbf{Principio de inducción fuerte(completo), \(PIF\)}

Sea \(P(n)\) un enunciado en el dominio de los números naturales.\\
 i. \(P(0)\) es verdadera y \\
 ii. Para todo \(n\in \N\), Si \(P(1),...,P(n-1),P(n)\) son todas
verdaderas, entonces \(P(n+1)\) es verdadera

Entonces \(P(n)\) es verdadera parar todo \(n \in \N\)

\textbf{\(\bf {PIC \Rightarrow PI}\)}

El principio de inducción completo implica el principio de inducción

El enunciado \(C\) se deduce a partir del enunciado \(A\), y esto a su
vez puede deducir al enunciado \(C\) a partir del enunciado \(A\) con
alguna otra hipótesis adicional \(B\).

\[(A\Rightarrow C) \Rightarrow (A\and B \Rightarrow C) \equiv \text{Tautología}\]

Donde

\[A= P(n),\quad B=P(0)\and P(1) \and ... \and P(n-1) \quad \text y \quad C = P(n+1)\]

Es decir, se puede usar el \(PI\) para demostrar \(PIC\). Supongamos que
asumimos como cierto el \(PI\) y queremos demostrar \(PIC\).

Sea \(P(n) \in \N\) que satisface \emph{(i)} y \emph{(ii)}.

Queremos demostrar que \(P(n)\) es verdadero para todo \(n \in \N\),
usando solo \(PI\).

Para esto creamos otro enunciado \(Q(n)\) tal que

\[Q(n) = P(0)\and P(1) \and ... \and P(n)\]

Tenemos que \(Q(1) = P(1)\) y por \emph{(ii)}

\[Q(n) \Rightarrow P(n+1)\]

Pero sabemos que

\[Q(n) =P(1) \and P(2) \and...\and P(n)\]

y también conocemos \(P(n+1)\)

Entonces conocemos

\[P(1)\and P(2) \and ... \and P(n) \and P(n+1) = Q(n+1)\]

Entonces \(Q(1)\) se cumple y para todo \(n\),
\(Q(n) \Rightarrow Q(n+1)\)

Entonce por el principio de inducción, \(Q(n)\) se cumple para toda
\(n\) por tanto, \(P(n)\) se cumple para todo \(n\)

\end{document}
